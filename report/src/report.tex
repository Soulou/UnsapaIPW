\documentclass{report}
\usepackage[utf8x]{inputenc}
\usepackage[T1]{fontenc}
\usepackage[francais]{babel}
\usepackage{graphicx}
\usepackage{hyperref}
\author{Léo Unbekandt - Guillaume Paran - Lucas Saurel}
\date{Mai 2012}
\title{Projet web - UnsapaIPW}

\begin{document}

  \maketitle
  \tableofcontents

  \section*{Introduction}
  \addcontentsline{toc}{section}{Introduction}

  \section{Fonctionnalités de l'application}
  L'application web que nous avons réalisée est un système de gestion d'examens. On distingue trois catégories d'utilisateurs:
  \begin{itemize}
        \item{Les étudiants : ils consultent les examens auxquels leur promotion est inscrite et déposent leurs compositions.}
        \item{Les responsables d'examens : ils créent des examens et notent les étudiants.}
        \item{L'administrateur : il gère l'ensemble des utilisateurs, des promotions et des examens.}
      \end{itemize}

    Un nouvel utilisateur a la possibilité de s'inscrire sur le site en tant qu'étudiant en indiquant sa promotion. S'il a des modifications à apporter sur les informations qu'il a fournies lors de son inscription, il peut se rendre sur la page "Mon profil" dédiée.
    
    Nous avons également mis en place une page de statistiques accessible à tous. Elle fournie différentes informations sur les examens terminés, dont les moyennes par examen. Les données sont regroupées par onglets désignant les différentes promotions.
  
  \section{Organisation de l'équipe}
  	Afin de travailler le plus efficacement possible, nous avons décidé d'utiliser l'outil collaboratif Github. Ceci nous a permis d'avoir chacun notre dépot des sources du projet et de soumettre nos modifications de façon simple.
  	
  	Concernant la répartition des tâches, nous avons fonctionné en déposant régulièrement des requêtes sur Github et chacun était libre de mettre en place les fonctionnalités qu'il voulait.
  	
  \section{Schéma de données}
    Le sujet nous imposait d'utiliser plusieurs entitées auxquelles nous avons ajouté un certain nombre d'attributs.

    \begin{figure}
      \caption{Schéma relationnel}
      \includegraphics[width=0.9\textwidth]{./data.png}
    \end{figure}

    \begin{figure}
      \caption{Modèle physique de base de données}
      \includegraphics[width=0.9\textwidth]{./db.png}
    \end{figure}

    \subsection{User}
      En plus des champs prérequis :
      \begin{itemize}
        \item{Prénom}
        \item{Nom}
        \item{Adresse}
        \item{Code postal}
        \item{Ville}
        \item{Adresse e-mail}
        \item{Numéro de téléphone}
      \end{itemize}

      Le UserBundle nous rajoute toute la partie nécessaire à la gestion de
      l'utilisateur côté serveur.
      \begin{itemize}
        \item{Le mot de passe $\Rightarrow$ chiffré en sha1 dans la BDD}
        \item{Les rôles $\Rightarrow$ Pour la gestion des permissions}
        \item{Nom d'utilisateur}
        \item{Expiration du compte, confirmation par email etc.}
      \end{itemize}

      Et enfin nous avons ajouté un champ pour notre besoin qui est une
      promotion, en effet, on considère qu'un étudiant appartient à une
      promotion précise.

    \subsection{Promo}
      Les promotions correspondent à un regroupement d'utilisateur, dans le cas d'utilisation présent, tous les étudiants d'une même année 
      appartiennent à une promotion unique.

    \subsection{Exam}
      Un examen représente une épreuve définie par un enseignant. L'examen 
			comprend les propriétés suivantes :
			\begin{itemize}
				\item{Titre}
				\item{Descrption}
				\item{Promotion}
				\item{Coefficient}
				\item{Date limite}
				\item{Responsable : Un user responsable de TD}
			\end{itemize}
			Par défaut, lors de la création de l'examen, tous les étudiants de la promotion sélectionnée sont concernés par l'examen, mais il est également possible de modifier au cas par cas si un étudiant est affecté par l'examen.
    \subsection{Record}
			L'entité Record fait le lien entre les étudiants et les examens. En effet dans un record on attribue pour un étudiant et un examen, la note, ainsi que le document rendu (Fichier pdf ou word).

			On a donc :
			\begin{itemize}
				\item{Étudiant}
				\item{Examen}
				\item{Note}
				\item{Document}
			\end{itemize}

			Ainsi un Record avec la note et le document NULL, correspond au fait qu'un étudiant doit effectuer un rendu pour un tel examen. Quand il rend quelque chose, on peuple le champ 'Document'. Et ensuite quand l'examen est terminé le responsable de l'examen peut associer une note.


  \section{Réalisation Technique}
    \subsection{UserBundle}
      Nous avons utilisé un Bundle d'extension nommé : 
      \href{https://github.com/FriendsOfSymfony/FOSUserBundle}{UserBundle}.
      
      Ce bundle constitue la brique applicative qui permet d'effectuer toutes les actions nécessaires à presque tous les projets, c'est-à-dire :
      \begin{itemize}
        \item{Inscription}
        \item{Authentification}
        \item{Connexion}
        \item{Déconnexion}
        \item{Gestion du profil}
        \item{Changement de mot de passe}
      \end{itemize}
      Nous évitant du travail laborieux qui ne sert pas sur le plan métier de l'application.

    \subsection{Les tests avec PHPUnit}
			Nous avons principalement testé les différentes entités. Grâce à PHPUnit nous avons généré la couverture en tests de notre code.

			\href{http://ares-ensiie.eu/~unbekandt2011/UnsapaIPW/cov}{Couverture des tests}
    \subsection{La documentation développeur avec phpDocumentor}
			Tout notre code a été documenté en utilisant phpDocumentor. On peut trouvé cette documentation à l'adresse :

			\href{http://ares-ensiie.eu/~unbekandt2011/UnsapaIPW/doc}{Documentation développeur}

  \section*{Conclusion}
  \addcontentsline{toc}{section}{Conclusion}

\end{document}
